

\section{Introduction}

The codes given here are a free implementation of the Generalized
Minimal Residual Method (GMRES) in c++. The method is fully described
by Saad\cite{sparseIterative} and Kelley\cite{iterativeMethods}. The
code given here is based on the pseudo code given by Barrett
\textit{et al}\cite{templates}. The codes were adapted to c++ after
examining the matlab codes by Burkardt\cite{BurkardtCode}. This
specific implementation makes use of restarts, and it was most
influenced by the code available from the United States' National
Institute of Standards and Technology\cite{imlCode}.

The GMRES algorithm is defined in the file GMRES.h. The algorithm is
given in a template function, named {\tt GMRES}. The subroutine
assumes that three classes are defined that include several specific
operators and functions. There is an additional subroutine called by
{\tt GMRES} named {\tt Update}. This subroutine is used to generate an
updated approximation based on the Krylov subspace generated within
the {\tt GMRES} subroutine.

There is an additional set of subroutines that are used within the
{\tt GMRES} subroutine. These routines are defined in the files {\tt
  util.h} and {\tt util.cpp}. The file {\tt util.h} includes the file
{\tt util.cpp}. Both files are assumed to be in the same directory as
the {\tt GMRES.h} file.

In this overview the call for the GMRES routine is first given, and
then the three required classes are stated in turn. First the {\tt
  Operation} class is discussed, then the {\tt Approximation} class,
and finally the {\tt Preconditioner} class is discussed. Each class
must implement specific operations and define a given set of
methods. The expectations are given within each section.


\section{Calling the GMRES Subroutine}

The GMRES routine is named {\tt GMRES}, and the definition for the
method is given in Listing \ref{listing:GMRESDefinition}. There are
seven parameters for the function. The first four parameters are
pointers to the respective classes. The next three parameters dictate
the behaviour and limits of the GMRES implementation and include the
number of vectors in the Krylov subspace, the number of restarts, and
the tolerance respectively.


\begin{lstlisting}[caption={The definition for the GMRES routine.},
                   basicstyle=\scriptsize,
                   label=listing:GMRESDefinition]
template<class Operation,class Approximation,class Preconditioner,class Double>
int GMRES
(Operation* linearization, //!< Performs the linearization of the PDE on the approximation.
 Approximation* solution,  //!< The approximation to the linear system. (and initial estimate!)
 Approximation* rhs,       //!< the right hand side of the equation to solve.
 Preconditioner* precond,  //!< The preconditioner used for the linear system.
 int krylovDimension,      //!< The number of vectors to generate in the Krylov subspace.
 int numberRestarts,       //!< Number of times to repeat the GMRES iterations.
 Double  tolerance         //!< How small the residual should be to terminate the GMRES iterations.
 )
\end{lstlisting}

The basic idea is that an approximation to a linear system is to be
generated. The system can be represented by
\begin{eqnarray}
  \label{eqn:basicLinearSystem}
  L \vec{x} & = & \vec{b}.
\end{eqnarray}
The parameters in the subroutine correspond to the following symbols
in equation (\ref{eqn:basicLinearSystem}):
\begin{eqnarray*}
  L & = & {\tt linearization}, \\
  \vec{x} & = & {\tt solution}, \\
  \vec{b} & = & {\tt rhs}.
\end{eqnarray*}

The parameter in the {\tt Operation} class, {\tt linearization}, is
used to implement the action of the matrix multiplied by a vector. The
vector is given by an object in the {\tt Approximation} class which
includes the initial estimate, {\tt solution}, and the right hand
side, {\tt rhs}. The variable {\tt solution} is updated, and it is
changed by calling the routine. 

The final class, {\tt Preconditioner}, is used to implement a
preconditioner for the system. The assumption is that the
preconditioned system is in the form
\begin{eqnarray*}
  P^{-1} L \vec{x} & = & P^{-1} \vec{b}.
\end{eqnarray*}
The GMRES algorithm requires that the operator, {\tt precond} in the
GMRES routine, be used to generate a sequence of vectors given by
\begin{eqnarray*}
  \vec{v}_{n+1} & = & L \hat{v}_n,
\end{eqnarray*}
where $\hat{v}_n$ is a normalized vector that is orthogonal to the
previous vectors generated. The {\tt GMRES} subroutine calculates this
under the assumption that the system to solve is given by the system
\begin{eqnarray}
  \label{eqn:preconditionerSolver}
  P \vec{v}_{n+1} & = & L \hat{v}_n,
\end{eqnarray}
where the routine solves for the vector $\vec{v}_{n+1}$.


\begin{lstlisting}[caption={The definition for the Update routine.},
                   basicstyle=\scriptsize,
                   label=listing:updateDefinition]
template <class Approximation, class Double >
void 
Update
(Double **H,       //<! The upper diagonal matrix constructed in the GMRES routine.
 Approximation *x, //<! The current approximation to the linear system.
 Double *s,        //<! The vector e_1 that has been multiplied by the Givens rotations.
 std::vector<Approximation> *v,  //<! The orthogonal basis vectors for the Krylov subspace.
 int dimension)    //<! The number of vectors in the basis for the Krylov subspace.
)
\end{lstlisting}

Finally, the file includes an additional subroutine, {\tt
  Update}. This subroutine is used to perform the back-solve and
update to determine the next approximation to the linear system. This
is used within the {\tt GMRES} subroutine and is not expected to be
called by another routine.


\section{The Operation Class}

The first class examined is the {\tt Operation} class. This class is
used to perform the actions equivalent to a matrix multiply. It is
assumed that this is in the form of a right multiply, i.e.
\begin{eqnarray*}
  L \cdot \vec{x}.
\end{eqnarray*}

\begin{lstlisting}[caption={An example of the multiply operation
    defined for the Operation class.},
                   basicstyle=\scriptsize,
                   label=listing:operationMultiply]
Approximation Operation::operator*(class Approximation vector)
{
  	Solution result;
    ...
    return(result);
}
\end{lstlisting}

The {\tt Operation} class must have a number of operations defined. In
particular the class should have the multiply operator defined. An
example of the required definition is given in Listing
\ref{listing:operationMultiply}.



\section{The Approximation Class}

The {\tt Approximation} class is used to keep track of the
approximation to the solution to the linear system. It is the class
used to store $\vec{x}$ as well as $\vec{b}$ in equation
(\ref{eqn:basicLinearSystem}). It is assumed that the {\tt
  Approximation} class has a number of methods and operations defined,
and the complete list is given in Table
\ref{tab:approximationOperations}.

\begin{table}
  \centering
  \begin{tabular}{l|l}
    Methods & Operations \\ \hline
    {\tt norm}     & $+$ \\
    {\tt getN}     & $-$ \\
    2 constructors & $+=$ \\
    axpy           & $=$ \\
                   & $*$
  \end{tabular}
  \caption{The methods and operations that must be defined for the {\tt Approximation} class.}
  \label{tab:approximationOperations}
\end{table}

Examples of the basic form for the definitions are given in Listing
\ref{listing:approximationOperations}. Note the types of the
arguments. For example, the addition operator is for the sum of two
objects from the {\tt Approximation} class while the multiplication
operator is for an object from the {\tt Approximation} class
multiplied on the right by a real valued variable.


\begin{lstlisting}[caption={An example of the operations that must be
    defined for the {\tt Approximation} class.},
                   basicstyle=\scriptsize,
                   label=listing:approximationOperations]

Approximation Approximation::operator+(const Approximation& vector)
{
	Approximation result(this->getN());
    ...
	return(result);
}


Approximation Approximation::operator-(const Approximation& vector)
{
	Approximation result(this->getN());
    ...
	return(result);
}

Approximation Approximation::operator+=(const Approximation& vector)
{
    ...
	return(*this);
}

Approximation Approximation::operator=(const Approximation& vector)
{
    ...
	return(*this);
}

Approximation Approximation::operator*(const double& value)
{
	Approximation result(this->getN());
    ...
	return(result);
}

\end{lstlisting}

There are four methods that must be defined for the {\tt
  Approximation} class. Examples of the definitions are shown in
Listing \ref{listing:approximationMethods}. Note that there are two
constructors that must be called. The first is a constructor that
requires the number of entries to allocate in the approximation. The
second is a constructor that makes a copy of the {\tt Approximation}
object passed to it.

Also note that the {\tt axpy} method is a method that adds a scalar
multiplied by another {\tt Approximation} object to the current
object. This method is used in the modified Gramm-Schmidt
orthogonalization of the vectors that make up the Krylov subspace.

\begin{lstlisting}[caption={An example of the methods that must be
    defined for the {\tt Approximation} class.},
                   basicstyle=\scriptsize,
                   label=listing:approximationMethods]

Approximation::Approximation(int size)
{
  ...
}

Approximation::Approximation(const Approximation& oldCopy)
{
  ...
}

double Approximation::norm(const Approximation& v1)
{
	double norm = v1.getEntry(0)*v1.getEntry(0);
	int lupe;
	for(lupe=v1.getN();lupe>0;--lupe)
		{
			norm += v1.getEntry(lupe)*v1.getEntry(lupe);
		}
	return(sqrt(norm));
}


int Approximation::getN() const
{
  return(N);
}

void Approximation::axpy(Approximation* vector,
					double multiplier)
{
	int lupe;
	for(lupe=getN();lupe>=0;--lupe)
		{
			setEntry(getEntry(lupe)+multiplier*vector->getEntry(lupe),lupe);
		}
}

\end{lstlisting}



\section{The Preconditioner Class}

The final class is the {\tt Preconditioner} class. This class only
requires one method, the {\tt solve} method. This method is used to
solve the system given in equation
(\ref{eqn:preconditionerSolver}). An example of the definition is
given in Listing \ref{listing:preconditionerMethods}. Notice that it
returns an object from the {\tt Approximation} class.


\begin{lstlisting}[caption={An example of the solve method that must be
    defined for the {\tt Preconditioner} class.},
                   basicstyle=\scriptsize,
                   label=listing:preconditionerMethods]

Approximation Preconditioner::solve(const Approximation &current)
{
	Approximation multiplied(current);

	// Perform the forward solve to invert the first part of the
	// Cholesky decomposition.
	int lupe;
	intermediate[0] = current.getEntry(0)/vector[0][0];
	for(lupe=1;lupe<=getN();++lupe)
		intermediate[lupe] = 
			(current.getEntry(lupe)-vector[lupe][1]*intermediate[lupe-1])
			/vector[lupe][0];

	// Perform the backwards solve for the Cholesky decomposition.
	multiplied(getN()) = intermediate[getN()]/vector[getN()][0];
	for(lupe=getN()-1;lupe>=0;--lupe)
		multiplied(lupe) = (intermediate[lupe]-multiplied(lupe+1)*vector[lupe+1][1])
			/vector[lupe][0];

	// The previous solves wiped out the boundary conditions. Restore
	// the left and right boundaru condition before sending the result
	// back.
	multiplied(0) = current.getEntry(0);
	multiplied(getN()) = current.getEntry(getN());
	return(multiplied);
}


\end{lstlisting}



%%% Local Variables: 
%%% mode: latex
%%% TeX-master: "doxygenHeader"
%%% End: 
